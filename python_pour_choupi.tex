\documentclass[a4paper, 11pt, oneside, draft]{book}
\usepackage[latin1]{inputenc}
\usepackage[francais]{babel}

\title{Le Python expliqué à mon Choupi}
\author{Choupi}
\date{\today}

\begin{document}

\maketitle

\frontmatter
\chapter{Introduction}{Le pourquoi du comment}
Pourquoi s'emmerder à apprendre le Python? Je suis sûr que tu te pose la question mon amour.
Et même si je te sais naturellement curieuse et que tu aime apprendre de nouvelles chose, pourquoi
se faire chier à apprendre un langage de programmation? Surtout sachant que ça demande un peu
[beaucoup?] de boulot. Après tout si ça te passionnait tant que ça, tu en aurai fait ton boulot [Ebawai],
et d'ailleurs y'a des gens qui en ont fait leur boulot et qui seront là pour faire les trucs relou
d'informaticien à ta place.

\mainmatter

\part[Théorie]{Moi je veux aller vivre en théorie, parcequ'en théorie tout se passe bien}
J'imagine que tu meurt d'impatience de te lancer dans le concret, de taper des lignes de codes sur un écran noir comme les hackers
dans les films américain. Mais pour que tu comprenne un minimum ce que tu es en train de faire, on va pas y couper il te faut
un poil de théorie. Après rassure toi, je vais condenser au maximum et ne garder que ce qui est absolument indispensable. Le but
c'est pas d'y passer 3 mois non plus.

Ce chapitre va se découper en trois parties, une partie de maths (je vois d'ici ton visage réjoui à cette perspective), une partie
d'informatique théorique et une dernière partie qui est un peu moins théorique et qui portera sur tout ce qui est environnement
de développement.

Je préfère te prévenir tout de suite, je metterais des exercices dans ces chapitres. Et oui, carrément des exercices que je te regarderais
faire avec un air sévère (attention à la punition en cas de mauvaise réponse!). Plus sérieusement, ces exercices seront courts
et faciles, le but étant que tu pratique les choses 2-3 fois afin de te faire la main. C'est en forgeant qu'on devient forgeron, toussa toussa...


\chapter[Les Maths (avec un grand 'M')]{}
blablabla

\backmatter

\tableofcontents


\end{document}
